% Created 2023-02-14 Tue 13:06
% Intended LaTeX compiler: pdflatex
\documentclass[11pt]{article}
\usepackage[utf8]{inputenc}
\usepackage{lmodern}
\usepackage[T1]{fontenc}
\usepackage[svgnames]{xcolor}
\usepackage{graphicx}
\usepackage{longtable}
\usepackage{float}
\usepackage{wrapfig}
\usepackage{rotating}
\usepackage[normalem]{ulem}
\usepackage{amsmath}
\usepackage{textcomp}
\usepackage{grffile}
\usepackage{marvosym}
\usepackage{wasysym}
\usepackage{amssymb}
\usepackage{amsmath}
\usepackage{svg}
\usepackage[theorems, skins]{tcolorbox}
\usepackage[version=3]{mhchem}
\usepackage{url}
\usepackage{minted}
\usepackage[strings]{underscore}
\usepackage{hyperref}
\usepackage{attachfile}
\usepackage{breakurl}
\usepackage{newuli}
\usepackage{uli-german-paragraphs}
\usepackage{natbib}
\usepackage{natmove}
\author{Ulrich G Wortmann}
\date{\today}
\title{World Ocean Atlas}
\begin{document}


\subsection{Goal}
\label{sec:org5466588}

This assignment is meant to help you write your notes on phosphate, nitrate, and oxygen cycling. Specifically, this assignment will allow you to create the necessary figures, and challenges you to make you own observations and descriptions. In other words, each subsection will yield an independent Obsidian note. If you already have notes, on these topics, please add to them.

\subsection{Primary Production}
\label{sec:org231136b}

Photosynthesis uses oxidized carbon (CO\textsubscript{2}) to create reduced carbon (organic matter OM). This process additionally requires nitrate and phosphate, resulting in organic matter with fairly uniform C:N:P ratios, the so called Redfield ratio (106:16:1). In addition, photosynthesis creates oxygen as a byproduct. 

Settling organic matter transports these elements from the surface ocean into the deep ocean. This process is termed export production and/or biological carbon pump.

Organic matter decomposes while sinking into the deep ocean, a 
process that consumes oxygen, and liberates phosphate and nitrate. This process is referred to as remineralization or respiration.  Ocean mixing via the thermohaline circulation returns those nutrients back to the surface ocean.

[BROKEN LINK: my\_images/OM\_cycling.pdf]

\subsubsection{Phosphate}
\label{sec:orgbe8faf7}
In the left pane you should see a file called \texttt{woa\_phosphate.ipynb}. Please open it into a new tab. If you execute the code in that file, you should see a map of the phosphate concentrations in the ocean.

\begin{enumerate}
\item Name two regions where organic matter export removes the majority of phosphate (2 pts)
\item Name two regions where ocean mixing returns phosphate to the surface ocean (2 pts).
\item In line 12, change the value of \texttt{z\_min\_max} to \texttt{[0, 4, 0.2]}, and set the ocean depth to 2000 (line 6), and rerun the code.  Can you confirm that the phosphate values in the deep ocean are higher than in the surface ocean?
\item Speculate why the phosphate concentrations in the Pacific deep ocean are higher than in the Atlantic deep ocean. Hint: Look at your note for the thermohaline circulation.
\item Where does the phosphate in the come from? Name two minerals.
\item How much of the particulate organic phosphor export from the surface ocean to the deep oceans is buried and how much is remineralized?
\end{enumerate}

\subsubsection{Oxygen}
\label{sec:org3867446}

Unlike phosphate, oxygen is also affected by temperature since the solubility of oxygen in water depends on the water temperature. 

[BROKEN LINK: my\_images/o-vs-t.pdf]

Use the \texttt{Oxygen.ipynb} and \texttt{Temperature.ipynb} notebooks to generate maps for oxygen and temperature. 

\begin{enumerate}
\item Compare both maps, and describe if there is a (rough) correlation between both datasets. Describe in words how the data correlates. 1 pt
\item Create an oxygen concentration map for the oxygen concentration at 800 meter below sealevel (mbsl). Compare the values against the map for 50 mbsl, and describe the difference in words. Explain what is going on. 4 pts
\item Photosynthesis creates oxygen, and remineralization consumes oxygen. However, not all organic matter is remineralized. As such there is a constant surplus of oxygen, which would lead to ever increasing oxygen concentrations. Which process balances the surplus oxygen from photosynthetic production?
\end{enumerate}

At this point, it may be necessary to delete some of the data files that were downloaded. In the left pane, select all files with the \texttt{*.nc} extension and delete them.

\subsubsection{Nitrate}
\label{sec:org87c6adc}

Generate a map for the nitrate concentration at 50 mbsl. Compare the values against the phosphate map for 50 mbsl.  
\begin{enumerate}
\item Do they show a similar spatial distribution?
\item Both, nitrate and phosphate a nutrients that  are required for photosynthesis.  However, nitrate concentrations are about and order of magnitude higher than phosphate concentrations. Explain why 4pts
\item Since not all organic matter is remineralized, there is a constant loss of nitrate. Unlike phosphate, nitrate loss occurs through an additional process. What is the name of this process? 2pts
\item What is the relevance of this process for atmospheric dinitrogen gas? 2pts
\item Which two source of nitrate and/or ammonium compensate for the loss of nitrate in the ocean? 2pts
\end{enumerate}

\section{Submission instructions}
\label{sec:org52952a8}

Submit the following Obsidian notes as pdf file on Quercus

\begin{itemize}
\item Phosphate cycling
\item Oxygen cycling
\item Nitrate cycling
\end{itemize}

\subsection{Additional Marking pointers}
\label{sec:orgd1b330a}

\begin{itemize}
\item Write your note in such a way  images are scaled to a with of 800 pixel. 2pts
\item Make sure your notes link to other notes where appropriate (i.e., Phosphate, would be a link to your Phosphate note etc.) 2 pts
\end{itemize}
\end{document}